\documentclass[10pt,a4paper]{article}

\input{AEDmacros}
\usepackage{caratula} % Version modificada para usar las macros de algo1 de ~> https://github.com/bcardiff/dc-tex


\titulo{Trabajo práctico 1}
\subtitulo{Especificación y WP}

\fecha{21/4/2024}

\materia{AED}
\grupo{Grupo IATOGYSWWBKAFJVCRWKR}

\integrante{Calo, Agustín}{390/23}{caloagustin4@gmail.com}
\integrante{Seri, Rafael Nicolás}{362/23}{rafaelnicoseri@gmail.com}
\integrante{Pintos Oliveira, Sol María Marcela}{428/23}{solpintosoliveira@gmail.com}
\integrante{Páez Torrico, Santiago}{713/23}{santiagopaez122@gmail.com}
% Pongan cuantos integrantes quieran

% Declaramos donde van a estar las figuras
% No es obligatorio, pero suele ser comodo
\graphicspath{{../static/}}

\begin{document}









\maketitle
\section{Especificación}
\subsection{redistribucionDeLosFrutos}
\begin{proc}{redistribucionDeLosFrutos}{\In recursos: \TLista{\float}, \In cooperan: \TLista{\bool}}{\TLista{\float}}
	\requiere {|recursos| = |cooperan|}
	\requiere {todosPositivos(recursos)}
	\asegura{|res|=|recursos|}
    \asegura{\paraTodo[unalinea]{i}{\ent}{0\leq i<|res| \implicaLuego if\ cooperan[i]\ then\ res[i] = totalARepartir(recursos, cooperan)\ else\ res[i] = recursos[i] + totalARepartir(recursos, cooperan)\ fi}}
\end{proc}

\aux{totalARepartir}{recursos: \TLista{\float}, cooperan: \TLista{\bool}}{\float}{\\
(\sum_{i = 0}^{|recursos| - 1}\ if\ cooperan[i]\ then\ recursos[i]\ else\ 0\ fi)\ /\ |recursos|}


\subsection{trayectoriaDeLosFrutosIndividualesALargoPlazo}
\begin{proc}{trayectoriaDeLosFrutosIndividualesALargoPlazo}{\Inout trayectorias: \TLista{\TLista{\float}}, \In cooperan: \TLista{\bool}, \In apuestas: \TLista{\TLista{\float}}, \In pagos: \TLista{\TLista{\float}}, \In eventos: \TLista{\TLista{\float}}} {}
	\requiere {-}
	\asegura{-}
\end{proc}


\subsection{trayectoriaExtrañaEscalera}
\begin{proc}{trayectoriaExtrañaEscalera}{\In trayectorias: \TLista{\float}} {\bool}
	\requiere {|trayectoria| > 0}
	\asegura{res = True \iff }
\end{proc}

\pred{maximoLocal}{s: \TLista{\float}}{\existe[unalinea]{i}{\ent}{0<i<|s|-1 \yLuego (s[i]>s[i+1] \land s[i]>s[i-1])}}

\subsection{individuoDecideSiCooperarONo}
\begin{proc}{individuoDecideSiCooperarONo}{\In individuo: \nat, \In recursos: \TLista{\float}, \Inout cooperan: \TLista{\bool}, \In apuestas: \TLista{\TLista{\float}}, \In pagos: \TLista{\TLista{\float}}, \In eventos: \TLista{\TLista{\nat}}} {}
	\requiere {-}
	\asegura{-}
\end{proc}


\subsection{individuoActualizaApuesta}
\begin{proc}{individuoActualizaApuesta}{\In individuo: \nat, \In recursos: \TLista{\float}, \In cooperan: \TLista{\bool}, \Inout apuestas: \TLista{\TLista{\bool}}, \In pagos: \TLista{\TLista{\float}}, \In eventos: \TLista{\TLista{\nat}}} {}
	\requiere {-}
	\asegura{-}
\end{proc}

\subsection*{Auxiliares y predicados globales}
\pred{todosPositivos}{s: \TLista{\float}}{\paraTodo[unalinea]{i}{\ent}{0\leq i<|s| \implicaLuego s[i] > 0}}






\section{Demostraciones de correctitud}

Demostrar que la siguiente especificación es correcta respecto de su implementación.

La función \textbf{frutoDelTrabajoPuramenteIndividual} calcula, para el ejemplo de apuestas al juego de cara o seca, cuánto se ganaría si se juega completamente solo. Se contempla que el evento True es cuando sale cara.

\begin{proc}{frutoDelTrabajoPuramenteIndividual}{\In recurso: \float, \In apuesta: \ensuremath{\langle s: \float, c: \float \rangle}, \In pago: \ensuremath{\langle s: \float, c: \float \rangle}, \In eventos: \TLista {\bool}, \Out res: \float } {}
	\requiere {apuesta_c + apuesta_s = 1 \land pago_c > 0 \land pago_s > 0 \land apuesta_c > 0 \land apuesta_s > 0 \land recurso > 0}
	\asegura{res = recurso(apuesta_c pago_c)^{\#apariciones(eventos,T)} (apuesta_s pago_s)^{\#apariciones(eventos,F )}}
\end{proc}

Donde \#apariciones(eventos, T) es el auxiliar utilizado en la teórica, y \#(eventos, T) es su abreviación.
\begin{lstlisting}
	res := recursos
	i := 0
	while (i < |eventos|) do
		if eventos[i] then
			res := (res * apuesta.c) * pago.c
		else
			res := (res * apuesta.s) * pago.s
		endif
		i := i + 1
	endwhile
	\end{lstlisting}

\end{document}