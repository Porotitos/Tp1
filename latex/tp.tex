\documentclass[10pt,a4paper]{article}

\usepackage[spanish,activeacute,es-tabla]{babel}
\usepackage[utf8]{inputenc}
\usepackage{ifthen}
\usepackage{listings}
\usepackage{dsfont}
\usepackage{subcaption}
\usepackage{amsmath}
\usepackage[strict]{changepage}
\usepackage[top=1cm,bottom=2cm,left=1cm,right=1cm]{geometry}%
\usepackage{color}%
\newcommand{\tocarEspacios}{%
	\addtolength{\leftskip}{3em}%
	\setlength{\parindent}{0em}%
}

% Especificacion de procs

\newcommand{\In}{\textsf{in }}
\newcommand{\Out}{\textsf{out }}
\newcommand{\Inout}{\textsf{inout }}

\newcommand{\encabezadoDeProc}[4]{%
	% Ponemos la palabrita problema en tt
	%  \noindent%
	{\normalfont\bfseries\ttfamily proc}%
	% Ponemos el nombre del problema
	\ %
	{\normalfont\ttfamily #2}%
	\
	% Ponemos los parametros
	(#3)%
	\ifthenelse{\equal{#4}{}}{}{%
		% Por ultimo, va el tipo del resultado
		\ : #4}
}

\newenvironment{proc}[4][res]{%
	
	% El parametro 1 (opcional) es el nombre del resultado
	% El parametro 2 es el nombre del problema
	% El parametro 3 son los parametros
	% El parametro 4 es el tipo del resultado
	% Preambulo del ambiente problema
	% Tenemos que definir los comandos requiere, asegura, modifica y aux
	\newcommand{\requiere}[2][]{%
		{\normalfont\bfseries\ttfamily requiere}%
		\ifthenelse{\equal{##1}{}}{}{\ {\normalfont\ttfamily ##1} :}\ %
		\{\ensuremath{##2}\}%
		{\normalfont\bfseries\,\par}%
	}
	\newcommand{\asegura}[2][]{%
		{\normalfont\bfseries\ttfamily asegura}%
		\ifthenelse{\equal{##1}{}}{}{\ {\normalfont\ttfamily ##1} :}\
		\{\ensuremath{##2}\}%
		{\normalfont\bfseries\,\par}%
	}
	\renewcommand{\aux}[4]{%
		{\normalfont\bfseries\ttfamily aux\ }%
		{\normalfont\ttfamily ##1}%
		\ifthenelse{\equal{##2}{}}{}{\ (##2)}\ : ##3\, = \ensuremath{##4}%
		{\normalfont\bfseries\,;\par}%
	}
	\renewcommand{\pred}[3]{%
		{\normalfont\bfseries\ttfamily pred }%
		{\normalfont\ttfamily ##1}%
		\ifthenelse{\equal{##2}{}}{}{\ (##2) }%
		\{%
		\begin{adjustwidth}{+5em}{}
			\ensuremath{##3}
		\end{adjustwidth}
		\}%
		{\normalfont\bfseries\,\par}%
	}
	
	\newcommand{\res}{#1}
	\vspace{1ex}
	\noindent
	\encabezadoDeProc{#1}{#2}{#3}{#4}
	% Abrimos la llave
	\par%
	\tocarEspacios
}
{
	% Cerramos la llave
	\vspace{1ex}
}

\newcommand{\aux}[4]{%
	{\normalfont\bfseries\ttfamily\noindent aux\ }%
	{\normalfont\ttfamily #1}%
	\ifthenelse{\equal{#2}{}}{}{\ (#2)}\ : #3\, = \ensuremath{#4}%
	{\normalfont\bfseries\,;\par}%
}

\newcommand{\pred}[3]{%
	{\normalfont\bfseries\ttfamily\noindent pred }%
	{\normalfont\ttfamily #1}%
	\ifthenelse{\equal{#2}{}}{}{\ (#2) }%
	\{%
	\begin{adjustwidth}{+2em}{}
		\ensuremath{#3}
	\end{adjustwidth}
	\}%
	{\normalfont\bfseries\,\par}%
}

% Tipos

\newcommand{\nat}{\ensuremath{\mathds{N}}}
\newcommand{\ent}{\ensuremath{\mathds{Z}}}
\newcommand{\float}{\ensuremath{\mathds{R}}}
\newcommand{\bool}{\ensuremath{\mathsf{Bool}}}
\newcommand{\cha}{\ensuremath{\mathsf{Char}}}
\newcommand{\str}{\ensuremath{\mathsf{String}}}

% Logica

\newcommand{\True}{\ensuremath{\mathrm{true}}}
\newcommand{\False}{\ensuremath{\mathrm{false}}}
\newcommand{\Then}{\ensuremath{\rightarrow}}
\newcommand{\Iff}{\ensuremath{\leftrightarrow}}
\newcommand{\implica}{\ensuremath{\longrightarrow}}
\newcommand{\IfThenElse}[3]{\ensuremath{\mathsf{if}\ #1\ \mathsf{then}\ #2\ \mathsf{else}\ #3\ \mathsf{fi}}}
\newcommand{\yLuego}{\land _L}
\newcommand{\oLuego}{\lor _L}
\newcommand{\implicaLuego}{\implica _L}

\newcommand{\cuantificador}[5]{%
	\ensuremath{(#2 #3: #4)\ (%
		\ifthenelse{\equal{#1}{unalinea}}{
			#5
		}{
			$ % exiting math mode
			\begin{adjustwidth}{+2em}{}
				$#5$%
			\end{adjustwidth}%
			$ % entering math mode
		}
		)}
}

\newcommand{\existe}[4][]{%
	\cuantificador{#1}{\exists}{#2}{#3}{#4}
}
\newcommand{\paraTodo}[4][]{%
	\cuantificador{#1}{\forall}{#2}{#3}{#4}
}

%listas

\newcommand{\TLista}[1]{\ensuremath{seq \langle #1\rangle}}
\newcommand{\lvacia}{\ensuremath{[\ ]}}
\newcommand{\lv}{\ensuremath{[\ ]}}
\newcommand{\longitud}[1]{\ensuremath{|#1|}}
\newcommand{\cons}[1]{\ensuremath{\mathsf{addFirst}}(#1)}
\newcommand{\indice}[1]{\ensuremath{\mathsf{indice}}(#1)}
\newcommand{\conc}[1]{\ensuremath{\mathsf{concat}}(#1)}
\newcommand{\cab}[1]{\ensuremath{\mathsf{head}}(#1)}
\newcommand{\cola}[1]{\ensuremath{\mathsf{tail}}(#1)}
\newcommand{\sub}[1]{\ensuremath{\mathsf{subseq}}(#1)}
\newcommand{\en}[1]{\ensuremath{\mathsf{en}}(#1)}
\newcommand{\cuenta}[2]{\mathsf{cuenta}\ensuremath{(#1, #2)}}
\newcommand{\suma}[1]{\mathsf{suma}(#1)}
\newcommand{\twodots}{\ensuremath{\mathrm{..}}}
\newcommand{\masmas}{\ensuremath{++}}
\newcommand{\matriz}[1]{\TLista{\TLista{#1}}}
\newcommand{\seqchar}{\TLista{\cha}}

\renewcommand{\lstlistingname}{Código}
\lstset{% general command to set parameter(s)
	language=Java,
	morekeywords={endif, endwhile, skip},
	basewidth={0.47em,0.40em},
	columns=fixed, fontadjust, resetmargins, xrightmargin=5pt, xleftmargin=15pt,
	flexiblecolumns=false, tabsize=4, breaklines, breakatwhitespace=false, extendedchars=true,
	numbers=left, numberstyle=\tiny, stepnumber=1, numbersep=9pt,
	frame=l, framesep=3pt,
	captionpos=b,
}

\usepackage{caratula} % Version modificada para usar las macros de algo1 de ~> https://github.com/bcardiff/dc-tex
\usepackage{amssymb}
\titulo{Trabajo práctico 1}
\subtitulo{Especificación y WP}

\fecha{21/4/2024}

\materia{AED}
\grupo{Grupo IATOGYSWWBKAFJVCRWKR}

\integrante{Calo, Agustín}{390/23}{caloagustin4@gmail.com}
\integrante{Seri, Rafael Nicolás}{362/23}{rafaelnicoseri@gmail.com}
\integrante{Pintos Oliveira, Sol María Marcela}{428/23}{solpintosoliveira@gmail.com}
\integrante{Páez Torrico, Santiago}{713/23}{santiagopaez122@gmail.com}
% Pongan cuantos integrantes quieran

% Declaramos donde van a estar las figuras
% No es obligatorio, pero suele ser comodo
\graphicspath{{../static/}}


\begin{document}









\maketitle
\section{Especificación} 
\subsection{redistribucionDeLosFrutos}
\begin{proc}{redistribucionDeLosFrutos}{\In recursos: \TLista{\float}, \In cooperan: \TLista{\bool}}{\TLista{\float}}
	\requiere {|recursos|=|cooperan|}
	\requiere {todosPositivos(recursos)}
	\asegura{|res|=|recursos|}
    \asegura{\paraTodo[unalinea]{i}{\ent}{0\leq i<|res| \implicaLuego \IfThenElse{cooperan[i]}{res[i]=totalARepartir(recursos, cooperan)}{res[i]=recursos[i]+totalARepartir(recursos, cooperan)}}}
\end{proc}

\aux{totalARepartir}{recursos: \TLista{\float}, cooperan: \TLista{\bool}}{\float}{\\ 
(\sum_{i=0}^{|recursos|-1} \IfThenElse{cooperan[i]}{recursos[i]}{0}) / |recursos|}


\subsection{trayectoriaDeLosFrutosIndividualesALargoPlazo}
\begin{proc}{trayectoriaDeLosFrutosIndividualesALargoPlazo}{\Inout trayectorias: \TLista{\TLista{\float}}, \In cooperan: \TLista{\bool}, \In apuestas: \TLista{\TLista{\float}}, \In pagos: \TLista{\TLista{\float}}, \In eventos: \TLista{\TLista{\float}}} {}
	\requiere{trayectorias=old(trayectorias)}
	\requiere{|cooperan|=|pagos|=|apuestas|=|eventos|=|trayectorias|}
	\requiere{\paraTodo[unalinea]{i}{\ent}{0\leq i<|pagos| \implicaLuego todosPositivos(pagos[i]) \land todosPositivos(apuestas[i])}}
	\requiere{\paraTodo[unalinea]{i}{\ent}{0\leq i<|trayectoria| \implicaLuego trayectoria[i][0] > 0}}
	\requiere{\paraTodo[unalinea]{j}{\ent}{-1<j<|apuestas| \implicaLuego \sum_{i=0}^{|eventos[j]|-1} apuestas[j][i]=1}}
	\asegura{|trayectorias|=|old(trayectorias)|}
	\asegura{\paraTodo[unalinea]{i}{\ent}{0\leq i<|old(trayectorias)| \implicaLuego |trayectorias[i]|=|old(trayectorias)[i]|+|eventos[i]|}}
	\asegura{\paraTodo[unalinea]{i}{\ent}{0\leq i<|old(trayectorias)| \implicaLuego trayectorias[i][0] = old(trayectorias)[i][0]}}
	\asegura{\paraTodo[unalinea]{i}{\ent}{0\leq i<|old(trayectorias)| \implicaLuego \paraTodo[unalinea]{j}{\ent}{0\leq j<|eventos[i]| \implicaLuego trayectorias[i][j+1] = \IfThenElse{cooperan[i]}{distribucion(aporteIndividual(trayectorias, apuestas, pagos, eventos, cooperan, i, j))}{aporte$-$\\
	Individual(trayectorias, apuestas, pagos, eventos, i, j) + distribucion(trayectorias, apuestas, pagos, eventos, cooperan,\\ j)}}}} 
\end{proc}

\aux{distribucion}{trayectorias: \TLista{\TLista{\float}}, apuestas: \TLista{\TLista{\float}}, pagos: \TLista{\TLista{\float}}, eventos: \TLista{\TLista{\nat}}, cooperan: \TLista{Bool}, m: \nat}{\float}{\\
(\sum_{k=0}^{|cooperan|-1} \IfThenElse{cooperan[k]}{aporteIndividual(trayectorias,\ apuestas,\  pagos,\ eventos,\ k,\ m)}{0}) / |cooperan|}


\subsection{trayectoriaExtrañaEscalera}
\begin{proc}{trayectoriaExtrañaEscalera}{\In trayectorias: \TLista{\float}} {\bool}
	\requiere {|trayectoria|>0}
	\asegura{res=True \iff |trayectoria| = 1 \lor 
    (trayectoria[0]>trayectoria[1] \land \neg maximoLocal(trayectoria) \land trayectoria[|trayectoria|-1]<trayectoria[|trayectoria|-2]) \lor
    (trayectoria[|trayectoria|-1]>trayectoria[|trayectoria|-2] \land \neg maximoLocal(trayectoria) \land trayectoria[0]<trayectoria[1]) \lor
    \existe[unalinea]{i}{\ent}{0<i<|trayectoria|-1 \yLuego (trayectoria[i]>trayectoria[i+1] \land trayectoria[i]>trayectoria[i-1]) \land \paraTodo[unalinea]{i}{\ent}{0<j<|trayectoria|-1 \yLuego (trayectoria[j]>trayectoria[j+1] \land trayectoria[j]>trayectoria[j-1]) \implicaLuego j=i}}}
\end{proc}

\pred{maximoLocal}{s: \TLista{\float}}{\existe[unalinea]{i}{\ent}{0<i<|s|-1 \yLuego (s[i]>s[i+1] \land s[i]>s[i-1])}}


\subsection{individuoDecideSiCooperarONo}
\begin{proc}{individuoDecideSiCooperarONo}{\In individuo: \nat, \In recursos: \TLista{\float}, \Inout cooperan: \TLista{\bool}, \In apuestas: \TLista{\TLista{\float}}, \In pagos: \TLista{\TLista{\float}}, \In eventos: \TLista{\TLista{\nat}}} {}
	\requiere{cooperan=old(cooperan)}
   \requiere{|cooperan|=|recursos|=|apuestas|=|pagos|=|eventos|}
   \requiere{\paraTodo[unalinea]{i}{\ent}{0\leq i<|apuestas| \implicaLuego todosPositivos(recursos) \land todosPositivos(apuestas[i]) \land \\ todosPositivos(pagos[i])}}
   \requiere{0\leq individuo<|cooperan|}
   \asegura{|cooperan|=|old(cooperan)|}
   \asegura{\paraTodo[unalinea]{i}{\ent}{0\leq individuo<|cooperan| \land i\neq individuo \implicaLuego cooperan[i]=old(cooperan)[i]}}
   \asegura{\existe[unalinea]{s,p}{\TLista{\TLista{\float}}}{|s|=|p|=|cooperan| \land \paraTodo[unalinea]{n}{\ent}{0\leq n<|s| \yLuego |s[n]|= |p[n]|=(|eventos|+1) \land s[n][0]= p[n][0]=recursos[n] \land \paraTodo[unalinea]{k}{\ent}{0<k<|s[n]| \implicaLuego s[n][k]= (\IfThenElse{old(cooperan)[k] \lor k=individuo}{0}{\\aporteIndividual(s, apuestas, pagos, eventos, n, k)}) + distribucionCoop(s, apuestas, pagos, eventos, old(cooperan),\\ k, individuo) \land p[n][k]= (\IfThenElse{\neg(old(cooperan)[k]) \lor k=individuo}{aporteIndividual(p, apuestas, pagos, eventos,n, k)\\}{0}) + distribucionNoCoop(p, apuestas, pagos, eventos, old(cooperan), k, individuo)}} \yLuego cooperan[individuo]= p[individuo][|p[individuo]|-1] \leq s[individuo][|s[individuo]|-1]}}
\end{proc}

\aux{distribucionCoop}{trayectorias: \TLista{\TLista{\float}}, apuestas: \TLista{\TLista{\float}}, pagos: \TLista{\TLista{\float}}, eventos: \TLista{\TLista{\nat}}, cooperan: \TLista{\bool}, m: \nat, individuo: \nat}{\float}{\\(\sum_{k=0}^{|cooperan|-1} \IfThenElse{cooperan[k] \lor k=individuo}{aporteIndividual(trayectorias, apuestas,  pagos, eventos, k, m)\\}{0})/|cooperan|}
\aux{distribucionNoCoop}{trayectorias: \TLista{\TLista{\float}}, apuestas: \TLista{\TLista{\float}}, pagos: \TLista{\TLista{\float}}, eventos: \TLista{\TLista{\nat}}, cooperan: \TLista{\bool}, m: \nat, individuo: \nat}{\float}{\\(\sum_{k=0}^{|cooperan|-1} \IfThenElse{cooperan[k] \land k \neq individuo}{aporteIndividual(trayectorias, apuestas,  pagos, eventos, k, m)\\}{0})/|cooperan|}

\subsection{individuoActualizaApuesta}
\begin{proc}{individuoActualizaApuesta}{\In individuo: \nat, \In recursos: \TLista{\float}, \In cooperan: \TLista{\bool}, \Inout apuestas: \TLista{\TLista{\bool}}, \In pagos: \TLista{\TLista{\float}}, \In eventos: \TLista{\TLista{\nat}}} {}
	\requiere{apuestas = old(apuestas)}
   \requiere{|cooperan|=|recursos|=|apuestas|=|pagos|=|eventos|}
   \requiere{0\leq individuo <|cooperan|}
   \asegura{|apuestas| = |old(apuestas)|}
   \asegura{\paraTodo[unalinea]{i}{\ent}{-1<i<|apuestas| \implicaLuego |apuestas[i]| = |old(apuestas)[i]|}}
   \asegura{\existe[unalinea]{p, s}{\TLista{\TLista{\float}})(\exists mejorApuesta : \TLista{\float})((\forall posibleApuesta : \TLista{\float}}{ultElem(p, recursos, old(apuestas),\\
    pagos, cooperan, eventos, individuo, mejorApuesta) \geq ultElem(s, recursos, old(apuestas), pagos, cooperan, eventos,\\
	individuo, posibleApuesta) \implicaLuego apuestas[individuo]=mejorApuesta)}}
   
\end{proc}
\aux{ultElem}{t: \TLista{\TLista{\float}}, recursos: \TLista{\float}, apuestas: \TLista{\TLista{\float}}, pagos: \TLista{\TLista{\float}}, cooperan: \TLista{Bool}, eventos: \TLista{\nat}, individuo: \nat, posibleApuesta: \TLista{\nat}}{\float}{\\
\IfThenElse{trayectoriaPosible(t, recursos, apuestas, pagos, cooperan, eventos, individuo, posibleApuesta)}{t[individuo][|t[individuo]|-1]}{-1}}

\pred{trayectoriaPosible}{t: \TLista{\TLista{\float}}, recursos: \TLista{\float}, apuestas: \TLista{\TLista{\float}}, pagos: \TLista{\TLista{\float}}, cooperan: \TLista{Bool}, eventos: \TLista{\nat}, individuo: \nat, posibleApuesta: \TLista{\nat}}{|posibleApuesta| = |apuestas[individuo]| \land sumElem(posibleApuesta)=1 \land todosPositivos(posibleApuesta) |t|=|cooperan| \land \paraTodo[unalinea]{i}{\ent}{-1<i<|t| \yLuego |t[i]|=(|eventos|+1) \land t[i][0]=recursos[i] \land \paraTodo[unalinea]{j}{\ent}{0\leq j<|t[i]| \implicaLuego t[i][j+1] = (\IfThenElse{cooperan[i]}{0}{aporteIndDiferido(t[i], apuestas[i], pagos[i], eventos[i], i, j, individuo, posibleApuesta)})+ distribucionDiferida (t[i], apuestas[i], pagos[i], eventos[i], cooperan, i, j, individuo, posibleApuesta)}}}

\aux{aporteIndDiferido}{trayectoria: \TLista{\float}, apuestas: \TLista{\float}, pagos: \TLista{\float}, eventos: \TLista{\nat}, k: \nat, m: \nat, individuo: \nat, apuestaInd: \TLista{\float}}{\float}{\\
\IfThenElse{k=individuo}{trayectorias[m] \cdot apuestaInd[eventos[m]] \cdot pagos[eventos[m]]}{trayectorias[m] \cdot apuestas[eventos[m]] \cdot pagos[eventos[m]]}}

\aux{distribucionDiferida}{trayectoria: \TLista{\float}, apuestas: \TLista{\float}, pagos: \TLista{\float}, eventos: \TLista{\nat}, cooperan: \TLista{Bool}, k: \nat, m: \nat, individuo: \nat, apuestaInd: \TLista{\float}}{\float}{\\
(\sum_{k=0}^{|cooperan|-1} \IfThenElse{cooperan[k]}{aporteIndDiferido(trayectorias, apuestas,  pagos, eventos, k, m, individuo) }{0}) / |cooperan|}
\aux{sumElem}{s: \TLista{\float}}{\float}{\\ \sum_{i = 0}^{|s| - 1} s[i]}


\subsection*{Auxiliares y predicados globales}
\pred{todosPositivos}{s: \TLista{\float}}{\paraTodo[unalinea]{i}{\ent}{0\leq i<|s| \implicaLuego s[i]>0}}
\aux{aporteIndividual}{trayectorias: \TLista{\TLista{\float}}, apuestas: \TLista{\TLista{\float}}, pagos: \TLista{\TLista{\float}}, eventos: \TLista{\TLista{\nat}}, k: \nat, m: \nat}{\float}{trayectorias[k][m] \cdot apuestas[k][eventos[k][m]] \cdot pagos[k][eventos[k][m]]}


\section{Demostraciones de correctitud}

Demostrar que la siguiente especificación es correcta respecto de su implementación.

La función \textbf{frutoDelTrabajoPuramenteIndividual} calcula, para el ejemplo de apuestas al juego de cara o seca, cuánto se ganaría si se juega completamente solo. Se contempla que el evento True es cuando sale cara.

\begin{proc}{frutoDelTrabajoPuramenteIndividual}{\In recurso: \float, \In apuesta: \ensuremath{\langle s: \float, c: \float \rangle}, \In pago: \ensuremath{\langle s: \float, c: \float \rangle}, \In eventos: \TLista {\bool}, \Out res: \float } {}
	\requiere {apuesta_c + apuesta_s = 1 \land pago_c > 0 \land pago_s > 0 \land apuesta_c > 0 \land apuesta_s > 0 \land recurso > 0}
	\asegura{res = recurso(apuesta_c pago_c)^{\#apariciones(eventos,T)} (apuesta_s pago_s)^{\#apariciones(eventos,F )}}
\end{proc}

Donde \#apariciones(eventos, T) es el auxiliar utilizado en la teórica, y \#(eventos, T) es su abreviación.
\begin{lstlisting}
	res := recurso
	i := 0
	while (i < |eventos|) do
		if eventos[i] then
			res := (res * apuesta.c) * pago.c
		else
			res := (res * apuesta.s) * pago.s
		endif
		i := i + 1
	endwhile
	\end{lstlisting}
Decimos que un programa $S$ es correcto respecto de una especificación dada por una precondición $P$ y una postcondición $Q$, si siempre que el programa comienza en un estado que cumple $P$, el programa termina su ejecución, y en el estado final se cumple $Q$. Se denota con la siguiente tripla de Hoare: \\
\begin{center}
	$\{P\} S \{Q\}$
\end{center}
Como el programa que nos dan cuenta con ciclos, tenemos que probar que $\{Pre\}S1;while...;S3\{Post\}$ es válida, para ello tenemos que cumplir:
\begin{enumerate}
	\item $Pre \implicaLuego wp(S1, P_c)$
	\item $P_c \implicaLuego wp(while..., Q_c)$
	\item $Q_c \implicaLuego wp(S3, Post)$
\end{enumerate}
Por monotonía, esto nos permite demostrar que $Pre \implicaLuego wp(S1;while...;S3, Post)$ es verdadera. \\
Vamos a demostrar 1 y 2, para ello proponemos lo siguiente:
\begin{itemize}
	\item $P_c \equiv \{i = 0 \land res = recurso \land apuesta_c + apuesta_s = 1 \land pago_c > 0 \land pago_s > 0 \land apuesta_c > 0 \land apuesta_s > 0 \land recurso > 0\}$
	\item $Q_c \equiv \{res = recurso * \prod_{j = 0}^{|eventos| - 1} \IfThenElse{eventos[j]}{apuesta.c *pago.c}{apuesta.s * pago.s}\}$
	\item $I \equiv \{0 \leq i \leq |eventos| \land res = recurso * \prod_{j = 0}^{i - 1} \IfThenElse{eventos[j]}{apuesta.c *pago.c}{apuesta.s * pago.s}\}$
	\item $fv \equiv \{|eventos|-i \}$
	\item $B \equiv \{i < |eventos|\}$
\end{itemize}
\subsubsection*{Equivalencias entre $Q_c$ propuesto y $asegura$ dado:}
\begin{minipage}[t]{18cm}
	$asegura: res = recurso * (apuesta_c * pago_c)^{\#apariciones(eventos,T)} (apuesta_s * pago_s)^{\#apariciones(eventos,F )} \equiv recurso*(apuesta_c * pago_c)^{\sum_{i=0}^{|eventos| - 1} \IfThenElse{eventos[i] = T}{1}{0}}*(apuesta_s * pago_s)^{\sum_{i=0}^{|eventos| - 1} \IfThenElse{eventos[i] = F}{1}{0}}$ \\
	Por propiedad de potenciación: $x^{f(n)+f(m)} = x^{f(n)}*x^{f(m)}$ luego $x^{\sum_{i=0}^{n}f(i)} = \prod_{i = 0}^{n}x^{f(i)}$ \\
	$\equiv recurso * \prod_{j=0}^{|eventos|-1} \IfThenElse{eventos[j] = T}{(apuesta_c * pago_c)}{1} * \prod_{j=0}^{|eventos|-1} \IfThenElse{eventos[j] = F}{(apuesta_s * pago_s)}{1}$ \\
	Si $A={0\leq j<|eventos|-1: eventos[j]=T}$ y $B={0\leq j<|eventos|-1: eventos[j]=F}$ tengo que $A \cap B= \varnothing$ y como en las productorias el predicado del else es 1 (neutro multiplicativo), vale que si las juntamos queda: \\
	$asegura: res = recurso * \prod_{j = 0}^{|eventos| - 1} \IfThenElse{eventos[j]}{apuesta_c *pago_c}{apuesta_s * pago_s}\ \square$
\end{minipage}
\subsection*{Demostración $Pre \implicaLuego wp(S1, P_c)$}
\begin{minipage}[t]{18cm}
	Vamos a utilizar los axiomas 1 (asignación) y 2 (composicional) vistos en la teórica. \\
1. $Pre \implicaLuego wp(res:=recurso; i:=0, P_c) \stackrel{Axioma\ 2}{\equiv} wp(res:=recurso; wp(i:=0, P_c))$ \\
2. $wp(i:=0, \{i = 0 \land res = recurso \land apuesta_c + apuesta_s = 1 \land pago_c > 0 \land pago_s > 0 \land apuesta_c > 0 \land apuesta_s > 0 \land recurso > 0\}) \stackrel{Axioma\ 1}{\equiv} def(0) \yLuego 0 = 0 \land res = recurso \land apuesta_c + apuesta_s = 1 \land pago_c > 0 \land pago_s > 0 \land apuesta_c > 0 \land apuesta_s > 0 \land recurso > 0 \equiv res = recurso \land apuesta_c + apuesta_s = 1 \land pago_c > 0 \land pago_s > 0 \land apuesta_c > 0 \land apuesta_s > 0 \land recurso > 0 \equiv E_1$ \\
3. $wp(res:=recurso, E_1) \stackrel{Axioma\ 1}{\equiv} def(recurso) \yLuego recurso = recurso \land apuesta_c + apuesta_s = 1 \land pago_c > 0 \land pago_s > 0 \land apuesta_c > 0 \land apuesta_s > 0 \land recurso > 0 \equiv apuesta_c + apuesta_s = 1 \land pago_c > 0 \land pago_s > 0 \land apuesta_c > 0 \land apuesta_s > 0 \land recurso > 0 \equiv E_2$ \\
Y como $Pre \equiv E_2$ tenemos que $Pre \implica E_2\ \square$
\end{minipage}
\subsection*{Demostración $P_c \implicaLuego wp(while..., Q_c)$}
\begin{minipage}[t]{18cm}
Por Teorema del Invariante, vamos a mostrar que la tripla de Hoare:
\begin{center}
	$\{P_c\} while...\{Q_c\}$
\end{center}
es válida. Entonces tenemos siguiente:
\begin{enumerate}
	\item $P_c \implies I$
	\item $\{I\land B\}S\{I\}$
	\item $I \land \neg B \implies Q_c$
	\item $\{I\land v_{0}=fv\}S\{fv<v_{0}\}$
	\item $I\land fv \leq 0 \implies \neg B$
\end{enumerate}
Vamos a demostrar cada una de ellas.
\subsubsection*{Demostración para $P_c \implies I$:}
$P_c \equiv \{i = 0 \land res = recurso \land apuesta_c + apuesta_s = 1 \land pago_c > 0 \land pago_s > 0 \land apuesta_c > 0 \land apuesta_s > 0 \land recurso > 0\}$ \\
$I \equiv \{ 0 \leq i \leq |eventos| \land res = recurso * \prod_{j = 0}^{i - 1} \IfThenElse{eventos[j]}{apuesta.c *pago.c}{apuesta.s * pago.s}\}$ \\
Queremos probar que vale el invariante:
\begin{itemize}
	\item $0 \leq i \leq |eventos|$ \\ Vale, porque $i = 0$ y trivialmente sabemos que se encuentra entre 0 y la longitud de la secuencia.
	\item $res = recurso * \prod_{j = 0}^{i - 1} \IfThenElse{eventos[j]}{apuesta.c *pago.c}{apuesta.s * pago.s}$ \\ Tenemos que $i = 0$ y $res = recurso$ entonces, como $i = 0$ \\ 
	$res = recurso * \prod_{j = 0}^{i - 1} \IfThenElse{eventos[j]}{apuesta.c *pago.c}{apuesta.s * pago.s} = recurso * \prod_{j = 0}^{0 - 1} \IfThenElse{eventos[j]}{apuesta.c *pago.c}{apuesta.s * pago.s} = recurso * \prod_{j = 0}^{-1}$, que es un rango vacío, es decir, valdrá 1 y teníamos en $P_c$ que $res = recurso$ entonces, $recurso = recurso * 1 = recurso$
\end{itemize}
Y con esto se demuestra que $P_c \implies I$
\subsubsection*{Demostración para $I \land \neg B \implies Q_c$}
$I \equiv \{0 \leq i \leq |eventos| \land res = recurso * \prod_{j = 0}^{i - 1} \IfThenElse{eventos[j]}{apuesta.c *pago.c}{apuesta.s * pago.s}\}$ \\
$B \equiv \{i < |eventos|\}$ \\
$Q_c \equiv \{res = recurso * \prod_{j = 0}^{|eventos| - 1} \IfThenElse{eventos[j]}{apuesta.c *pago.c}{apuesta.s * pago.s}\}$
\end{minipage}
\end{document}